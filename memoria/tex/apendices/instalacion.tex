Como se ha dicho en otras secciones de esta memoria, la web interactiva, que es el producto software desarrollado en este proyecto, puede consultarse y utilizarse desde cualquier navegador desde la dirección \url{https://jantoniovr.github.io/Geometria-Fractal/}.

No obstante, por cuestiones de eficiencia recomendamos ejecutar el
proyecto en local. Explicamos el procedimiento en los siguientes
párrafos.

\begin{enumerate}
\def\labelenumi{\arabic{enumi}.}
\item
  Primero de todo necesitamos descargar el código del repositorio del proyecto en github, el cual se puede encontrar en \url{https://github.com/JAntonioVR/Geometria-Fractal}. Para ello, mediante una terminal nos situamos en el directorio que deseemos y clonamos el repositorio:

  \begin{lstlisting}[language=bash]
git clone git@github.com:JAntonioVR/Geometria-Fractal.git
cd Geometria-Fractal
  \end{lstlisting}

  Si no se dispone de \texttt{git}, en la pestaña \texttt{Code} se puede encontrar la posibilidad de descargar un \texttt{ZIP} con el código.

\item
  Es necesario lanzar un servidor web local. Para esto, se ofrecen algunas de las alternativas más sencillas. Si se dispone de
  \texttt{Python}, bastaría con ejecutar el comando:

  \begin{lstlisting}[language=bash]
python SimpleHTTPServer 8000 # Si se dispone de Python2
  \end{lstlisting}

  o

  \begin{lstlisting}[language=bash]
python http.server 8000 # Si se dispone de Python3 (recomendado)
  \end{lstlisting}
  
  Tras esto, abra su navegador en la URL \url{http://localhost:8000/} y encontrará la portada de la web.

  Otra alternativa para los usuarios del editor \texttt{Visual\ Studio\ Code} (VSCode) es la extensión \href{https://marketplace.visualstudio.com/items?itemName=ritwickdey.LiveServer}{\textcolor{blue}{\texttt{Live\ Server}}}, la cual con tan solo instalarla y pulsar el botón `Go Live' que aparecerá en la esquina inferior derecha lanza un servidor web en el directorio que se tenga abierto en el momento y abre una ventana en el navegador.

  Por último, en \href{https://nodejs.org/en/knowledge/HTTP/servers/how-to-create-a-HTTP-server/}{\textcolor{blue}{este enlace}} se explica la posibilidad de crear un servidor web utilizando \texttt{node.js}.

  \textbf{Nota:} Con Python se utiliza el puerto 8000, con `Live Server' el 5500 y con node.js el 8080. Sin embargo, es posible que por alguna razón esos puertos estén ocupados por otros procesos y aparezca algún error. Si esto ocurre simplemente utilice algún puerto distinto que esté disponible.

\item
  Independientemente de la opción que hayamos elegido para lanzar el servidor web, tan solo resta abrir el navegador que queramos e introducir la dirección \href{http://localhost:8000/}{\texttt{http://localhost:PPPP/}}, siendo \texttt{PPPP} el puerto que hayamos elegido finalmente.
\end{enumerate}