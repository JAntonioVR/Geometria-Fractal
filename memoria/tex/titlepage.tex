\begin{titlepage}
    \hspace{-1.5cm}
	\includegraphics[width=40mm]{img/LogoETSIIT.png}
	\includegraphics[width=40mm]{img/LogoFacultadCiencias.jpeg}
	\hfill
	\includegraphics[width=50mm]{img/LogoUGR.png}
	
	\noindent\begin{small} \sffamily
		\begin{minipage}{0.65\textwidth}
			Doble Grado en Ingeniería Informática y Matemáticas\\
			Curso 2021/2022\\
			Trabajo de Fin de Grado\\
		\end{minipage}
	\hrule
	\end{small}

	%Thesis title
	\vspace{1cm}
	{\LARGE\noindent \textbf{Fractales y Geometría Fractal} \par}
	\vspace{0.5cm}
	%Thesis subtitle
	{\Large\noindent Fractales, geometría fractal y aplicaciones en la ciencia. Visualización de fractales con Ray-Tracing \par}
	\vspace{2cm}
	%Author's name
	{\LARGE\noindent Autor: Juan Antonio Villegas Recio \par} 
	
	\vfill
		
	\hrule
	\vspace{0.3cm}
	
	\begin{table}[h!]
		\begin{footnotesize} \sffamily
			\begin{tabular}{p{0.21\textwidth}p{0.79\textwidth}}
				Autor: & Juan Antonio Villegas Recio \\
				Tutor de Matemáticas:    & Manuel Ruiz Galán, Catedrático de Universidad\\
				& Departamento de Matemática Aplicada, Universidad de Granada \\
				Tutor de Informática:      & Carlos Ureña Almagro, Profesor Titular de Universidad \\
				& Departamento de Lenguajes y Sistemas Informáticos, Universidad de Granada
			\end{tabular}
		\end{footnotesize}
	\end{table}
	
\end{titlepage}