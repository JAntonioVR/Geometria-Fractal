%
% ─── CAPITULO 6: VISUALIZACION DE FRACTALES 2D ──────────────────────────────────
%

En el capítulo \ref{chap:renderizacion} introdujimos el uso de WebGL como herramienta de renderizado de imágenes y estudiamos sus componentes, sin embargo, no olvidemos que nuestro objetivo es la visualización de fractales, donde la característica principal de los mismos es que no se pueden expresar a partir de un conjunto de vértices o líneas, sino que son curvas o superficies totalmente irregulares. Por tanto, a efectos prácticos, nuestro vertex shader tomará como entrada los vértices $(-1,-1)$, $(1,-1)$, $(1,1)$ y $(-1,1)$ y no aplicará ninguna transformación, pues ya están normalizados en el clip space (teniendo en cuenta que estaríamos visualizando un fragmento del plano $z=0$). Cabe en este momento aclarar que en el ámbito de herramientas de renderizado se utiliza el convenio de utilizar la coordenada $Y$ para la altura y la coordenada $Z$ para la profundidad. 

A partir de estos cuatro vértices, en el canvas se visualizarán dos triángulos que completarán la superficie completa del mismo. El vertex shader a partir de ahora será totalmente trivial, pues solo devolverá en la variable \verb|gl_Position| la misma posición que obtiene del buffer de posición.

\begin{lstlisting}
attribute vec2 a_Position;
void main() {
    gl_Position = vec4(a_Position.x, a_Position.y, 0.0, 1.0);
}
\end{lstlisting}

Mientras que, por su parte, el fragment shader podrá acceder a la posición (en coordenadas de dispositivo) del píxel que se está ejecutando mediante la variable \verb|gl_FragCoord| y a partir de estas coordenadas devolver un color en la variable \verb|gl_FragColor|. Es decir, estamos dibujando una escena completa, próximamente un fractal, en dos triángulos. Por ejemplo, las imágenes \ref{fig:julia-intro} y \ref{fig:mandelbrot-intro} son el resultado de esta metodología. 

\section{Objetivo}

Procedemos a explicar el objetivo principal de este objetivo y para el cual programaremos cada línea de código: Queremos desarrollar una página web interactiva, que cuente con un canvas donde se renderice el fractal que deseemos y, además, haya una serie de parámetros que se puedan controlar dinámicamente, de forma que conforme se cambia un parámetro el canvas modifica la imagen que está renderizando.

Queremos visualizar conjuntos de Julia $\mathsf{J}_c$ para distintos $c\in\C$, el conjunto de Mandelbrot, y las generalizaciones de los conjuntos de Julia y Mandelbrot ocasionadas si se itera la función $P_{c,N}(z)=z^N+c$ para distintos valores de $N\in\N$. Además, si revisitamos el algoritmo que utilizamos en la sección \ref{subsection:representacion-julia} para graficar en \textit{Mathematica} conjuntos de Julia y el que utilizamos en la sección \ref{subsection:representacion-mandelbrot} para visualizar el conjunto de Mandelbrot, podremos recordar que para aproximar qué puntos del plano complejo eran prisioneros o de escape fijábamos un número máximo de iteraciones $M$, tras las cuales se consideraba que un número $z_0\in\C$ era prisionero si la sucesión de los módulos de sus iteradas $\{P_{c,N}^n(z_0)\}$ no superaba el número de escape $e_c=\max\{2,|c|\}$ . Este valor $M$ también podría ser un parámetro modificable, para así poder ver dinámicamente cómo cambia la resolución cuando se cambia el número máximo de iteraciones.



\section{Estructurando el código}

Podemos usar como base el código utilizado para visualizar el cuadrado de colores, ya que nos puede venir bien su estructura para adaptar la misma a la renderización de fractales. Sin embargo, tiene una estructura muy procedural. Podemos mantener la misma arquitectura de forma que cambiando los elementos que sean necesarios y el código de los shaders podamos ver los fractales que deseemos, pero en ese caso la depuración se complicaría, el código es más difícil de leer y cuesta mucho añadir interactividad. Por este motivo, adaptaremos el código a un paradigma orientado a objetos, modularizando los distintos componentes, creando abstracciones de las herramientas que proporciona WebGL y siguiendo los \href{https://medium.com/backticks-tildes/the-s-o-l-i-d-principles-in-pictures-b34ce2f1e898}{principios SOLID}.

Esta separación se ha hecho siguiendo 
