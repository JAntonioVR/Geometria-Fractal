%%%%%%%%%%%%%%%%%%%%%%%%%%%%%%%%%%%%%%%%%%%%%%%%%%%
%% Capítulo 4: Sistemas de funciones iteradas
%%%%%%%%%%%%%%%%%%%%%%%%%%%%%%%%%%%%%%%%%%%%%%%%%%%

Como venimos viendo ya en los dos últimos capítulos, la iteración es una poderosa herramienta en la generación de imágenes fractales. Sin embargo, hasta ahora siempre nos estamos basando en buscar convergencia y velocidad de convergencia de sucesiones de iteradas de funciones complejas, las cuales evalúan un número complejo $z\in\C$ para devolver otro número complejo $f(z)\in\C$. Si recuperamos la identificación $z=x+y\cdot i\simeq (x,y)\in\R^2$ podemos ver las funciones complejas como campos vectoriales de $\R^2$, y si en lugar de aplicar una función a un único punto la aplicamos a un conjunto de puntos llegamos a la base de los \textit{Sistemas de Funciones Iteradas}, en adelante SFI. 

La matemática que explica los SFI se puede encontrar en el clásico libro \textit{Fractals Everywhere}\cite{Barnsley} de \textit{Michael Barnsley}. Por su parte, la geometría fractal nació en 1977 con la publicación de \textit{The fractal geometry of nature}\cite{alma991007242979704990} por parte de \textit{Benoit Mandelbrot}. En general, gracias a la geometría fractal y ayudándonos de los SFI podemos recrear resultados de imágenes y objetos con un nivel de detalle que la geometría euclídea no puede conseguir. Sin embargo, el problema inverso también es interesante: ¿es posible, a partir de un objeto, describirlo matemáticamente mediante SFI? Este es un área de la matemática aún abierta y en la que a día de hoy se continua trabajando. Uno de los resultados más conocidos en este ámbito es el \textit{teorema del collage}, del cual hablaremos más adelante. %TODO referenciar teorema y sección

\section{Transformaciones afines en el plano euclídeo y SFI}

Si nos restringimos al plano euclídeo $\R^2$ visto como espacio afín, previo a la definición formal de SFI necesitamos unas nociones sobre transformaciones afines y maneras de representarlas, ya que estas serán las que compongan fundalmente los SFI.

\begin{definicion}[Transformación afín]
    Una transformación $w:\R^2\longrightarrow\R^2$ de la forma
    \begin{equation}
        w(x_1,x_2)=(ax_1+bx_2+e,cx_1+dx_2+f) \ \ \forall (x_1,x_2)\in\R^2
    \end{equation}
    donde las constantes $a,b,c,d,e,f$ son números reales es denominada una \textbf{transformación afín} del plano euclídeo.
\end{definicion}

Una forma equivalente de denotar $w$ matricialmente es tomando la matriz $A=\begin{pmatrix}
    a & b \\
    c & d
\end{pmatrix}\in\mathcal{M}_2(\R)$ y el vector $b=\begin{pmatrix}
    e \\
    f
\end{pmatrix}$ y expresar $w$ como:

\begin{equation}
    w(x) =
    w\begin{pmatrix}
        x_1 \\
        x_2
    \end{pmatrix} = Ax + b = \begin{pmatrix}
        a & b \\
        c & d
    \end{pmatrix}\begin{pmatrix}
        x_1 \\
        x_2
    \end{pmatrix}+\begin{pmatrix}
        e \\
        f
    \end{pmatrix} \ \ \forall \ x = \begin{pmatrix}
        x_1 \\
        x_2
    \end{pmatrix}\in\R^2.
\end{equation}

La matriz $A$ también puede ser expresada de la siguiente forma:
\begin{equation}
    A = \begin{pmatrix}
        a & b \\
        c & d
    \end{pmatrix} = \begin{pmatrix}
        r\cos\alpha & -s\sin\beta \\
        r\sin\alpha & s\cos\beta
    \end{pmatrix}
\end{equation}
donde el par $(r,\alpha)$ son las coordenadas polares de $(a,c)$ y $(s,\beta+\frac \pi 2)$ son las coordenadas polares de $(b,d)$. De esta forma, una transformación afín puede verse representada por 6 números reales $r,s,\alpha,\beta,e,f$, de forma que:
\begin{itemize}
    \item $r,s$ representan una homotecia de razón $r$ en el eje $X$ y razón $s$ en el eje $Y$.
    \item $\alpha,\beta$ denotan una rotación de $\alpha$ radianes en la componente $X$ y $\beta$ en la componente $Y$.
    \item $e,f$ simbolizan una traslación de vector $b=\begin{pmatrix} e \\ f\end{pmatrix}$.
\end{itemize}

Nótese que la transformación lineal $x\mapsto Ax$ en $\R^2$ lleva un paralelogramo con un extremo en el origen en otro paralelogramo con un extremo en el origen, como consecuencia de la linealidad de las aplicaciones \textit{escalado} y \textit{rotación}. Por lo que la transformación afín $w(x)=Ax+b$ es una composición de la aplicación lineal representada por $A$, la cual transforma el espacio relativo al origen, y de la \textit{traslación} de vector $b=\begin{pmatrix} e \\ f\end{pmatrix}$. A continuación definimos un caso concreto de transformación afín más familiar.

\begin{definicion}
    Una transformación afín de $\R^2$ $w(x)=Ax+b$, para cualquier $b\in\R^2$ se denomina \textit{similitud} si la matriz $A$ tiene alguna de las siguientes formas:
    \begin{eqnarray}
        A = \begin{pmatrix}
            r\cos\alpha & -r\sin\alpha \\
            r\sin\alpha & r\cos\alpha
        \end{pmatrix}
        , &
        A = \begin{pmatrix}
            r\cos\alpha & r\sin\alpha \\
            r\sin\alpha & -r\cos\alpha
        \end{pmatrix}
    \end{eqnarray}
    para $r\not= 0,\ \ 0\leq\alpha<2\pi$. A $r$ se le llama \textit{razón de la homotecia} o factor de escala y a $\alpha$ se le llama \textit{ángulo de rotación}.
\end{definicion}

Nótese que en caso $\alpha=\pi,\beta=0$ se consigue una reflexión respecto al eje $Y$, y viceversa.

Para aclarar todos estos conceptos en la figura \ref{fig:ejemplos-ta} podemos ver cómo actúan distintas transformaciones lineales sobre una figura simple: el polígono formado al unir los vértices $(0,0), (1,0), (1.5, 0.5), (1,1)$ y $(0,1)$. Las transformaciones lineales vienen representadas en cada caso por una sextupla $w=(r,s,\alpha,\beta,e,f)$, teniendo cada elemento el significado ya definido.

\begin{figure}[ht]
    \centering
    \begin{tabular}{cc}
      \includegraphics[scale=0.65]{./img/C4/ejemplo-ta-1.png} &   \includegraphics[scale=0.6]{./img/C4/ejemplo-ta-2.png} \\
    (a) $w=(0.5, 0.5, \frac{\pi}{6}, \frac{\pi}{6}, 0,0)$ & (b) $(0.5, 0.5, \frac{\pi}{6}, 0,0,0)$ \\[6pt]
    \includegraphics[scale=0.6]{./img/C4/ejemplo-ta-3.png} &   \includegraphics[scale=0.75]{./img/C4/ejemplo-ta-4.png} \\
    (c) $(1,1,0,0,0.1,0.1)$ & (d) $(1,1,\pi,0,0,0)$ \\[6pt]
    \end{tabular}
    \caption{Ejemplos de aplicaciones de transformaciones lineales}
    \label{fig:ejemplos-ta}
  \end{figure}

En el caso de (a) observamos una similitud con $r=0.5$ y $\alpha=\frac{\pi}{6}$. En (b) además de un escalado uniforme de razón $r=s=0.5$ podemos ver el efecto que tiene una rotación (no uniforme) en $X$ de razón $\alpha=\frac{\pi}{6}$. Por su parte, (c) simplemente aplica una traslación mediante el vector $b=(0.1,0.1)^t$. Por último, en (d) vemos un caso de reflexión respecto al eje $Y$.

Un último detalle para la creación de SFI es la necesidad de que las transformaciones afines utilizadas sean \textit{aplicaciones contractivas}. Procedemos por tanto a definir un SFI:

\begin{definicion}[Sistema de Funciones Iteradas]
    \label{def:SFI}
    Un \textit{Sistema de Funciones Iteradas} se compone de un espacio métrico completo $(X,d)$ y de un conjunto finito de aplicaciones contractivas $w=\{w_i:i=1,\dots,n\}$.

    Se denomina \textit{constante de contractividad} del SFI a la mayor de las constantes de contractividad de las aplicaciones que lo forman, $s=\max\{s_i:i=1,\dots,n\}$, siendo $s_i$ la constante de contractividad de $w_i \ \ \forall i=1,\dots,n$.

    Dado un subconjunto $A\subseteq X$, la imagen de $A$ por medio de $w$ es definida como
    $$
    w(A)=\bigcup_{i=1}^n w_i(A).
    $$
\end{definicion}

Podemos utilizar como espacio métrico completo el plano euclídeo $\R^2$ y un conjunto finito de transformaciones lineales contractivas.

\begin{ejemplo}
    \label{ejemplo:sfi}
    Supongamos que tenemos un triángulo equilátero $T$ cuyos vértices son los puntos $(0,0),(1,0),(\frac{1}{2},\frac{\sqrt{3}}{2})$ y las transformaciones lineales
    \begin{eqnarray*}
        w_1 = \left(0.5,0.5,0,0,0,0\right) \\
        w_2 = \left(0.5,0.5,0,0,\frac 1 2,0\right) \\
        w_3 = \left(0.5,0.5,0,0,\frac 1 4,\frac{\sqrt{3}}{4}\right)
    \end{eqnarray*}
    Entonces en la imagen \ref{fig:ejemplo-sfi} podemos ver tanto $T$ como el resultado de aplicar el SFI $w$ a $T$.
    \begin{figure} [ht]
    \centering
    \includegraphics[scale = 0.6]{img/C4/ejemplo-sfi.png}
    \caption{Representación gráfica de $T$ y $w(T)$}
        \label{fig:ejemplo-sfi}
    \end{figure}
\end{ejemplo}

\section{Convergencia de SFI}
\label{section:convergencia-sfi}

Probablemente el resultado de aplicar el SFI $w$ a un triángulo equilátero que vemos en el ejemplo \ref{ejemplo:sfi} le recuerde al primer paso al generar el triángulo de Sierpinski, el cual vimos en la sección \ref{subsection:triangulo-Sierpinski}. De hecho, podemos volver a aplicar $w$ a $w(T)$, a $w(w(T))$, y así sucesivamente, de forma que iterando infinitamente $w$ en $T$, el resultado final que obtenemos es efectivamente el triángulo de Sierpinski \textbf{S}, véase de nuevo la imagen \ref{fig:triangulo-Sierpinski}.

Este es sólo un ejemplo, pero a lo largo de esta sección veremos que todo SFI converge a una figura, que denominamos el atractor del sistema, independientemente de la figura inicial. Véase en la figura \ref{fig:semillas-sfi} cómo incluso tomando como semilla una figura totalmente distinta al triángulo equilátero el resultado de la iteración es el mismo. Esto es de hecho una consecuencia del Teorema del punto fijo de Banach (teorema \ref{th:punto-fijo}).

\begin{figure} [ht]
    \centering
    \includegraphics[scale = 0.33]{img/C4/figura-X.png}
    \caption{Otra posible semilla para iterar el SFI}
        \label{fig:semilla-X}
\end{figure}

\begin{figure}[ht]
    \centering
    \begin{tabular}{cc}
      \includegraphics[scale=0.33]{./img/C4/triangulo-iterado.png} &   \includegraphics[scale=0.33]{./img/C4/figura-X-iterada.png} \\
    (a) Iterando un triángulo equilátero & (b) Iterando la figura \ref{fig:semilla-X} 
    \end{tabular}
    \caption{Resultado de iterar 8 veces $w$ con distintas figuras iniciales}
    \label{fig:semillas-sfi}
\end{figure}

\subsection{El Espacio de Fractales y la Métrica de Hausdorff}

Consideramos un espacio de Banach $(X,d)$\footnote{Recordamos que un espacio de Banach es un espacio métrico completo} y sea $\mathcal{H}(X)\subset\mathcal P (X)$ el conjunto de todos los subconjuntos compactos de $X$, el cual también se denomina \textit{espacio de fractales} de $X$. Dado un punto $x\in X$ y un subconjunto $A\in\mathcal H$ la distancia de un punto a un conjunto como
$$
d(A,B) = \inf\{d(x,b):b\in B\}=\min\{d(x,b):b\in B\},
$$
donde el ínfimo es un mínimo porque $A$ es compacto. Si tomamos dos compactos $A,B\in\mathcal H(X)$, entonces la distancia de $A$ a $B$ se define como:
$$
d(A,B)=\max(d(a,B):a\in A)
$$

El problema es que esta definición no nos basta para definir una distancia entre conjuntos, pues si $A\subset B$ tenemos que $d(A,B)=0$, pero si $b\in B\backslash A$, entonces $d(b,A)>0$, por lo que necesariamente $d(B,A)>0$. Y como sabemos, una auténtica distancia es conmutativa. Véase el contraejemplo de la figura \ref{fig:contraejemplo-distancia}. 
\begin{figure} [ht]
    \centering
    \includegraphics[scale = 0.4]{img/C4/no-distancia-hausdorff.png}
    \caption{Contraejemplo a la distancia entre conjuntos}
        \label{fig:contraejemplo-distancia}
\end{figure}

\begin{definicion}
    Dado un espacio de Banach $(X,d)$, en el espacio de fractales $\mathcal{H}(X)$ Se define la \textbf{distancia de Hausdorff} o \textbf{métrica de Hausdorff} como:
    $$
    h(d)(A,B)=\max\{d(A,B),d(B,A)\} \ \ \forall A,B\in\mathcal H(X).
    $$
\end{definicion}

En adelante denotaremos únicamente $h(\cdot,\cdot)$, omitiendo la dependencia de la distancia del espacio original.

Podemos comprobar en \cite[Sección 2.6]{Barnsley} que, en efecto, $h$ es una distancia. Además, en \cite[Sección 2.7]{Barnsley} se prueba que el espacio métrico $(\mathcal{H}(X), h(d))$ es completo.

\subsection{Aplicaciones contractivas en el espacio de fractales}

Consideramos un espacio métrico completo $(X,d)$ y su espacio de fractales dotado de la métrica de Hausdorff $(\mathcal H(X), h(d))$, que también es completo, y tomamos una aplicación contractiva $w:X\longrightarrow X$. Buscamos averiguar qué ocurre al iterar $w$ sobre $\mathcal{H}(X)$, siendo esta contractiva sobre $X$ y $\mathcal{H}(X)$ un espacio métrico completo. 

\begin{lema}
    \label{lema:contractivas-compactos}
    Sea $(X,d)$ un espacio métrico y $w:X\longrightarrow X$ una aplicación contractiva, entonces 
    $$w(\mathcal{H}(X))=\{w(A):A\in\mathcal{H}(X)\}\subseteq\mathcal{H}(X),$$
    es decir, la imagen por $w$ de todo compacto de $X$ es un conjunto compacto de $X$.
\end{lema}
\begin{proof}
    Sabemos que $w$ es continua en $X$, pues toda aplicación contractiva es lipschitziana y por tanto continua. Como la imagen de un conjunto compacto por una aplicación continua es un conjunto compacto, podemos afirmar que $w$ lleva elementos de $\mathcal{H}(X)$ a elementos de $\mathcal{H}(X)$.
\end{proof}

Ahora necesitamos alguna forma de construir aplicaciones contractivas en el espacio $(\mathcal{H}(X),h)$ a partir de aplicaciones contractivas en $(X,d)$. Gracias al siguiente lema comprobamos que la forma más natural es suficiente.

\begin{lema}
    Sea $(X,d)$ un espacio métrico y $w:X\longrightarrow X$ una aplicación contractiva con constante de Lipschitz $s$. Entonces $w:\mathcal{H}(X)\longrightarrow\mathcal{X}$, definida naturalmente como
    $$
    w(A)=\{w(a):a\in A\}\ \ \forall A\in\mathcal{H}(X)
    $$
    es una aplicación contractiva en $(\mathcal{H}(X),h)$.
\end{lema}
\begin{proof}
    Gracias al lema \ref{lema:contractivas-compactos} sabemos que la aplicación $w:\mathcal{H}(X)\longrightarrow\mathcal{H}(X)$ está bien definida, por lo que falta probar que en efecto es contractiva. Sean $A,B\in\mathcal{H}(X)$, tenemos que
    \begin{eqnarray*}
        d(w(A),w(B)) & = & \max\left\lbrace \min \left\lbrace d(w(a),w(b)): b\in B\right\rbrace : a\in A \right\rbrace \\
                     & \leq &  \max\left\lbrace \min \left\lbrace s\cdot d(a,b): b\in B\right\rbrace : a\in A \right\rbrace \\
                     & = & s\cdot d(A,B).
    \end{eqnarray*}
    Análogamente, $d(w(B),w(A))\leq s d(B,A)$. Por tanto
    \begin{eqnarray*}
        h(w(A),w(B)) & = & \max \{d(w(A),w(B)), d(w(B),w(A))\} \\
                     & \leq & s \max \{d(A,B), d(B,A))\} \\
                     & = & s\cdot h(A,B).
    \end{eqnarray*}
    Por lo que tenemos que $w$ es contractiva sobre $\mathcal{H}(X)$.
\end{proof}

Enunciamos una propiedad de la métrica de Hausdorff $h$ que nos hará falta dentro de poco.

\begin{lema}
    \label{lema:uniones}
    Sean $A,B,C,D\in\mathcal{H}(X)$, entonces 
    $$h(A\cup B, C\cup D)\leq\max\{h(A,C),h(B,D)\}$$.
\end{lema}
\begin{proof}
    La prueba se sigue de otra propiedad de la distancia $d$:
    \begin{eqnarray*}
    d(A\cup B,C) & = & \max\{d(x,C)\ \ \forall x\in A\cup B\} \\
                 & = & \max\left\lbrace \max\left\lbrace d(x,C):c\in A \right\rbrace, \max\left\lbrace d(x,C):c\in B \right\rbrace\right\rbrace \\
                 & = & \max\left\lbrace d(A,C),d(A,B) \right\rbrace.
    \end{eqnarray*}
    Y de esta igualdad se deduce la desigualdad pedida.
\end{proof}

Seguidamente presentamos un resultado que nos ayuda a construir aplicaciones contractivas a partir de no sólo una sino de varias, a través del cual enlazaremos, esta vez en un contexto más teórico y formal, con la definición de SFI como conjunto de aplicaciones contractivas en un espacio métrico completo.

\begin{proposicion}
    Sea $(X,d)$ un espacio métrico y sea $\{w_i:i=1,\dots,n\}$ un conjunto finito de aplicaciones contractivas $w_i:\mathcal{H}(X)\longrightarrow\mathcal{H}(X)$, cada una con constante de contractividad $s_i$. Definimos la aplicación $W:\mathcal{H}(X)\longrightarrow\mathcal{H}(X)$ como
    \begin{equation}
        \label{eqn:W}
        W(A) := \bigcup_{i=1}^n w(A) \ \ \forall A\in\mathcal{H}(X).
    \end{equation}
    Entonces $W$ es una aplicación contractiva en $\mathcal{H}(X)$ y su constante de contractividad es $s=\max\{s_i:i=1,\dots,n\}$.
\end{proposicion}
\begin{proof}
    Demostraremos el caso $n=2$, de forma que por inducción sería cierto para cualquier $n\leq 1$. Sean $A,B\in\mathcal{H}(X)$, tenemos que
    \begin{eqnarray*}
    h(W(A),W(B)) & = & h(w_1(A)\cup w_2(A), w_1(B)\cup w_2(B)) \\
                 & \leq & \max\{h(w_1(A),w_1(B)), h(w_2(A),w_2(B))\}\text{ (lema \ref{lema:uniones})} \\
                 & \leq & \max\{s_1\cdot h(A,B), s_2\cdot h(A,B) \} \\
                 & \leq & \max\{s_1,s_2\} h(A,B) \\
                 & = & s h(A,B)
    \end{eqnarray*}
    Lo cual completa la demostración.
\end{proof}

\subsection{El espacio de fractales y los SFI}

Hasta ahora hemos probado varios resultados que nos han servido para construir aplicaciones contractivas en el espacio de fractales de un espacio métrico a partir de un conjunto finito de aplicaciones contractivas. Si añadimos la hipótesis de la complitud, tendríamos con ese conjunto de aplicaciones contractivas un SFI, el cual podemos iterar bajo la seguridad de que no perdemos dicha contractividad. Por lo tanto, y recuperando la definición \ref{def:SFI}, ya podemos recopilar toda la información que tenemos y enunciar el siguiente teorema:

\begin{teorema}
    Consideramos el SFI formado por el espacio métrico completo $(X,d)$ y el conjunto $\{w_i:i=1\dots, N\}$ de aplicaciones contractivas, siendo $s$ la constante de contractividad del SFI. Consideramos también el espacio de fractales $(\mathcal{H}(X),h)$, donde definimos la aplicación $W:\mathcal{H}(X)\longrightarrow\mathcal{H}(X)$ como en (\ref{eqn:W}):
    $$
    W(A) := \bigcup_{i=1}^N w(A) \ \ \forall A\in\mathcal{H}(X).
    $$
    Entonces $W$ es una aplicación contractiva de constante $s$ y admite un único punto fijo $A^*\in\mathcal{H}(X)$, el cual está dado por 
    $$
    A^*=\lim_{n\rightarrow\infty} W^n(B) \ \ \forall B\in\mathcal{H}(X).
    $$

\end{teorema}
\begin{proof}
    Gracias a los resultados anteriores solo nos quedaría probar que $W$ admite un único punto fijo dado por la iteración infinita de $W$ e independientemente de qué $B\in\mathcal{H}(X)$ inicial se tome. Sin embargo, teniendo en cuenta que $W$ es contractiva y que $\mathcal{H}(X)$ es un espacio métrico completo, simplemente aplicando el teorema del punto fijo de Banach (teorema \ref{th:punto-fijo}) se tiene el resultado completo
\end{proof}

En resumen, tenemos que todo SFI admite un único punto fijo al que converge la iteración sucesiva de la aplicación $W$ construida a partir de su conjunto de aplicaciones contractivas. Además, esta convergencia está asegurada sea cual sea el conjunto inicial en el que se comienzan las iteradas. Como veníamos anunciando al inicio de esta sección, podemos ponerle nombre al punto fijo del SFI:

\begin{definicion}[Atractor]
    Dado un SFI, definimos que su único punto fijo como el \textbf{atractor} del SFI.    
\end{definicion}

Recordamos ahora el SFI del ejemplo \ref{ejemplo:sfi} y las imágenes \ref{fig:semillas-sfi}, que nos permitirían comprobar que el triángulo de Sierpinski es el atractor del SFI y que independientemente de la figura inicial (siempre y cuando sea un conjunto compacto de $\R^2$) la iteración conjunta de $\{w_1,w_2,w_3\}$ converge a $\mathbf{S}$ como resultado.

Ahora podemos aplicar esta teoría al espacio métrico completo $\R^2$ y a las transformaciones afines, que