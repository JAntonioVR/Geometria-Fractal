\documentclass[twoside,openright,11pt]{report}
\usepackage{import}
\usepackage{preamble}

\begin{document}

%%%%%%%%%%%%%%%%%%%%%%%%%%%%
%%% Cambiar 'Figura' por 'Imagen'
\renewcommand{\figurename}{Imagen}
\renewcommand{\listfigurename}{Lista de Imágenes}



%%%%%%%%%%%%%%%%%%%%%%%%%%%%
%%%% PORTADA
%%%%%%%%%%%%%%%%%%%%%%%%%%%%

\thispagestyle{empty}
\import{tex/}{titlepage.tex}

\newpage
%%%%%%%%%%%%%%%%%%%%%%%%%%%%

%\chapter*{Autoría}

%%%%%%%%%%%%%%%%%%%%%%%%%%%%
% AUTORÍA
%%%%%%%%%%%%%%%%%%%%%%%%%%%%

%\import{tex/}{autoria.tex}


%%%%%%%%%%%%%%%%%%%%%%%%%%%%
% ABSTRACT
%%%%%%%%%%%%%%%%%%%%%%%%%%%%

\chapter*{Resumen}

\import{tex/}{resumen.tex}

%%%%%%%%%%%%%%%%%%%%%%%%%%%%
% ABSTRACT
%%%%%%%%%%%%%%%%%%%%%%%%%%%%

\chapter*{Abstract}

\import{tex/}{abstract.tex}

%%%%%%%%%%%%%%%%%%%%%%%%%%%%
% ÍNDICE
%%%%%%%%%%%%%%%%%%%%%%%%%%%%

\tableofcontents
\setcounter{page}{1}
\pagenumbering{roman}
\thispagestyle{plain}


%%%%%%%%%%%%%%%%%%%%%%%%%%%%
% LISTA DE ABREVIATURAS
%%%%%%%%%%%%%%%%%%%%%%%%%%%%

\chapter*{Lista de Abreviaturas}

\import{tex/}{lista-abreviaturas.tex}
\addcontentsline{toc}{chapter}{Lista de Abreviaturas}

%%%%%%%%%%%%%%%%%%%%%%%%%%%%
% LISTA DE IMÁGENES
%%%%%%%%%%%%%%%%%%%%%%%%%%%%
\listoffigures
\thispagestyle{plain}
\addcontentsline{toc}{chapter}{Lista de Imágenes}

%%%%%%%%%%%%%%%%%%%%%%%%%%%%
% LISTA DE TABLAS
%%%%%%%%%%%%%%%%%%%%%%%%%%%%
\listoftables
\thispagestyle{plain}
\addcontentsline{toc}{chapter}{Lista de Tablas}
% \addtocontents{toc}{\bigskip}

%%%%%%%%%%%%%%%%%%%%%%%%%%%%
%%%%%%%%%%%%%%%%%%%%%%%%%%%%
% INICIO DEL TRABAJO
%%%%%%%%%%%%%%%%%%%%%%%%%%%%
%%%%%%%%%%%%%%%%%%%%%%%%%%%%

\chapter*{Introducción}
\addcontentsline{toc}{chapter}{Introducción}
\setcounter{page}{1}
\pagenumbering{arabic}


\import{tex/capitulos/}{introduccion.tex}

\chapter*{Objetivos}
\addcontentsline{toc}{chapter}{Objetivos}
\label{chap:Objetivos}

\import{tex/}{objetivos.tex}

\chapter{El concepto de \textit{fractal}.}
\label{chap:concepto}

\import{tex/capitulos/}{capitulo-1.tex}

\chapter{Iteración.}
\label{chap:iteracion}

\import{tex/capitulos/}{capitulo-2.tex}


\chapter{Conjuntos de Julia y Mandelbrot.}
\label{chap:Julia-Mandelbrot}

\import{tex/capitulos/}{capitulo-3.tex}

\chapter{Sistemas de Funciones Iteradas.}
\label{chap:SFI}

\import{tex/capitulos/}{capitulo-4.tex}

\chapter*{Desarrollo del Software: Planificación y presupuesto}
\label{chap:planificacion-presupuesto}

\import{tex/capitulos/}{planificacion-presupuestos.tex}

\chapter*{Desarrollo del Software: Análisis y diseño}
\label{chap:analisis-diseno}

\import{tex/capitulos/}{analisis-diseno.tex}

\chapter{Introducción a las herramientas de visualización}
\label{chap:visualizacion}

\import{tex/capitulos/}{capitulo-5.tex}

\chapter{Visualización de fractales en 2D.}
\label{chap:fractales-2D}

\import{tex/capitulos/}{capitulo-6.tex}

\chapter{Introducción al \textit{Ray-Tracing}.}
\label{chap:ray-tracing}

\import{tex/capitulos/}{capitulo-7.tex}

\chapter{Visualización de fractales en 3D.}
\label{chap:fractales-3D}

\import{tex/capitulos/}{capitulo-8.tex}

\chapter*{Conclusiones}
\addcontentsline{toc}{chapter}{Conclusiones}
\label{chap:conclusiones}

\import{tex/}{conclusiones.tex}

%%%%%%%%%%%%%%%%%%%%%%%%%%%%
% REFERENCIAS
%%%%%%%%%%%%%%%%%%%%%%%%%%%%

\import{tex/}{bibliography.tex}
%%\bibliographystyle{plain}
%%\bibliography{references}

\begin{comment}
\nocite{*}

\bibliographystyle{apalike} % We choose the "plain" reference style
\bibliography{references} % Entries are in the refs.bib file

%\nocite{Gerald, Falconer, Hurewicz-Wallman, Wagon, Apuntes-AMI-Paya, Ostrowski, Atkinson, John-Milnor, agness-scott, Sandra-Snyder, Evan-Dummit, Bandt, Foroutan, cgdirector, computer-graphics, wikipedia-webgl, MDN-2, MDN-3, MDN-4, MDN-5, learn-opengl, khronos, HTML5rocks, renderingcontextdoc, RT-que-es, shadows, 3D-SDFs, quaternion-product, romanescu, rayo, menger, calima, curvas-nivel}

\nocite{Gerald}
\nocite{Falconer}
\nocite{Hurewicz-Wallman}
\nocite{Wagon}
\nocite{Apuntes-AMI-Paya}
\nocite{Ostrowski}
\nocite{Atkinson}
\nocite{John-Milnor}

\nocite{agness-scott}
\nocite{Sandra-Snyder}
\nocite{Evan-Dummit}
\nocite{Bandt}
\nocite{Foroutan}
\nocite{cgdirector}
\nocite{computer-graphics}
\nocite{wikipedia-webgl}

\nocite{MDN-2}
\nocite{MDN-3}
\nocite{MDN-4}
\nocite{MDN-5}
\nocite{learn-opengl}
\nocite{khronos}
\nocite{HTML5rocks}
\nocite{renderingcontextdoc}



\defbibfilter{papers}{
  type=article or
  type=book or 
  type=misc or 
  type=inproceedings or 
  type=incollection or 
  type=report 
}

\printbibliography[
heading=bibintoc, filter=papers,
title={Libros y artículos}
] %Prints the entire bibliography with the title "Whole bibliography"

\printbibliography[
heading=bibintoc, type=online,
title={Enlaces a la web}
] %Prints the entire bibliography with the title "Whole bibliography"

\defbibfilter{multimedia}{
  type=image or
  type=video
}

\printbibliography[
heading=bibintoc, filter=multimedia,
title={Multimedia}
]

\end{comment}

%%%%%%%%%%%%%%%%%%%%%%%%%%%%
% APENDICES
%%%%%%%%%%%%%%%%%%%%%%%%%%%%

\appendix
\cleardoublepage
% \addtocontents{toc}{\bigskip}
\addcontentsline{toc}{part}{Appendices}

%% OPTIONAL - Max 10-15 pages in total

\chapter{Documentación del código JavaScript}
\label{appendix:javascript}

\import{tex/apendices/}{javascript-doc.tex}


\chapter{Another Appendix}

\end{document}
